\documentclass[12pt, a4 paper]{article}
% Set target color model to RGB
\usepackage[inner=2.0cm,outer=2.0cm,top=2.5cm,bottom=2.5cm]{geometry}
\usepackage{setspace}
\usepackage[rgb]{xcolor}
\usepackage{environ}
\usepackage{verbatim}
\usepackage{subcaption}
\usepackage{outlines}
\usepackage{amsgen,amsmath,amstext,amsbsy,amsopn,tikz,amssymb,tkz-linknodes}
\usepackage{fancyhdr}
\usepackage{pgfplots}
\usepackage{mathtools}
\usepackage[colorlinks=true, urlcolor=blue,  linkcolor=blue, citecolor=blue]{hyperref}
\usepackage[colorinlistoftodos]{todonotes}
\usepackage{rotating}

\linespread{1.6} % Double Line Spacing
\usetikzlibrary{arrows.meta,intersections,calc}


\hypersetup{%
pdfauthor={Vignesh Ravibaskar},%
pdfcreator={PDFLaTeX},%
pdfproducer={PDFLaTeX},%
}
\usepackage{enumitem}
\setlist[enumerate,1]{label=\textbf{\arabic*}}
\setlist[enumerate,2]{label=\textbf{({\alph*})}}
\setlist[enumerate,3]{label=\textbf{({\roman*})}}

\title{Functions}
\author{Derek, Vignesh}
\date{2020}

\newcommand{\comm}[1]{}
% \NewEnviron{answer}{\vspace{3mm} \\ \color{blue} {\BODY} \color{black}}
\NewEnviron{answer}{\color{blue} \comm{\BODY} \color{black}} % Use this method to hide all answers

\begin{document}

\maketitle

\textbf{FUNCTIONS [60 Marks]}
\begin{outline}[enumerate]
 \1 The function f is defined by \[\textrm{f}:x \mapsto \left|\ln{(3x)+1}\right|, x>a.\] %Question 1

 \2 Find the smallest value of $a$ such that \textrm{f}$^{-1}$ exists.\hfill[1]
 \begin{answer}
  Solving $\textrm{f}(x)=0$, we get $x=\dfrac{1}{3\mathrm{e}}$. Thus, the smallest value of $a$, $a_\textrm{min}=\dfrac{1}{3\mathrm{e}}$
 \end{answer}

 \2 Using the domain of f found in \textbf{(a)}, sketch the graph of $y=\textrm{f}(x)$ and $y=\textrm{f}^{-1}(x)$ on the same diagram. \hfill[4]
 \begin{answer}
  To get $\textrm{f}^{-1}(x)$, we make $y$ the subject of the equation $x=|\ln{(3y)+1}|$ so we get $\textrm{f}^{-1}(x)=\dfrac{\mathrm{e}^{x-1}}{3}$ \\
  \vspace{3mm}
  \color{black}
  \begin{tikzpicture}
   \begin{axis}[
     axis lines = center,
     xlabel = $x$,
     ylabel = $y$,
     xmin = 0, xmax=5,
     ymin = 0, ymax=5,
     legend pos=outer north east
    ]
    \addplot [
     domain=1/(3*e):5,
     samples=100,
     color=blue,
    ]
    {abs(ln(3*x)+1)};
    \addlegendentry{$\textrm{f}(x)=\left|\ln{(3x)+1}\right|$}
    \addplot [
     domain=0:5,
     samples=100,
     color=red,
    ]
    {(e^(x-1))/3};
    \addlegendentry{$\textrm{f}^{-1}(x)=\frac{\mathrm{e}^{x-1}}{3}$}
   \end{axis}
  \end{tikzpicture}
 \end{answer}


 \1 The function f is defined by \[\textrm{f}:x \mapsto \sqrt{a^2-x^2}+k,\;0< x <a.\] %Question 2

 \2 Given that f is \emph{self-inverse}, meaning that $\textrm{f}(x)=\textrm{f}^{-1}(x)$, state the conditions that must be fulfilled by $a$ and $k$. \hfill[2]
 \begin{answer}
  Let us make $x$ the subject of the equation $y=\sqrt{a^2-x^2}+k$
  \begin{align*}
   y       & =\sqrt{a^2-x^2}+k    \\
   (y-k)^2 & = a^2-x^2            \\
   x       & = \sqrt{a^2-(y-k)^2}
  \end{align*}
  It is evident that the equation is \emph{self-inverse} for $a\in\mathbb{R}^+, k=0$
 \end{answer}

 \2 Hence, find the exact value of $x$ such that $\textrm{f}^{-1}\textrm{f}^{-1}(x)=\textrm{f}\textrm{f}\textrm{f}(x)$.\hfill[3]
 \begin{answer}
  Since $\textrm{f}(x)=\textrm{f}^{-1}(x)$, $\textrm{f}^{-1}\textrm{f}^{-1}(x)=\textrm{f}\textrm{f}\textrm{f}(x)$ can be found by solving $x=\mathrm{f}x$
  \begin{align*}
   x          & =\sqrt{a^2-x^2}       \\
   \implies x & =\sqrt{\frac{a^2}{2}}
  \end{align*}
 \end{answer}

 \1 The function f is defined by \[\textrm{f}:x \mapsto \frac{1}{2}\sqrt{4-x^2},\;0\leq x \leq2.\] %Question 3

 \2 Find $\textrm{f}^{-1}(x)$ and state the domain and range of $\textrm{f}^{-1}$.\hfill[2]
 \begin{answer}
  To find $\textrm{f}^{-1}(x)$, let us make $x$ the subject of $y=\frac{1}{2}\sqrt{4-x^2}$
  \begin{align*}
   y & =\frac{1}{2}\sqrt{4-x^2}   \\
     & \implies 4y^2=4-x^2        \\
     & \implies x = 2\sqrt{1-y^2}
  \end{align*}
  Thus, $\textrm{f}^{-1}(x)=2\sqrt{1-x^2}$. Further, we know that D$_{\textrm{f}^{-1}} = \textrm{R}_\textrm{f} = [0,1], \textrm{R}_{\textrm{f}^{-1}} = \textrm{D}_\textrm{f} = [0,2]$
 \end{answer}

 \2 State the domains and ranges of $\textrm{f}\textrm{f}^{-1}$ and $\textrm{f}^{-1}\textrm{f}$.\hfill[2]
 \begin{answer}
  D$_{\textrm{f}\textrm{f}^{-1}}$ = D$_{\textrm{f}^{-1}}$ = $[0,1]$, R$_{\textrm{f}\textrm{f}^{-1}} = [0,1]$ \\
  D$_{\textrm{f}^{-1}\textrm{f}}$ = D$_{\textrm{f}}$ = $[0,2]$, R$_{\textrm{f}^{-1}\textrm{f}}$= $[0,2]$
 \end{answer}

 \2 State the set of values of $x$ for which $\textrm{f}\textrm{f}^{-1}(x)=\textrm{f}^{-1}\textrm{f}(x)$.\hfill[1]
 \begin{answer}
  The set of values is $[0,1] \cap [0,2] = [0,1]$
 \end{answer}

 \2 Find the exact solution of $\textrm{f}\textrm{f}^{-1}\textrm{f}(x)=\frac{1}{2}$.\hfill[2]
 \begin{answer}
  Since $\textrm{f}^{-1}\textrm{f}(x)=x$, we can solve for $x$ in $\textrm{f}(x)=\frac{1}{2}$ which is equivalent to $\textrm{f}^{-1}(\frac{1}{2})=\sqrt{3}$
 \end{answer}

 \1 The function f is defined by \[\textrm{f}(x)=\frac{1}{1-x},\;x\in \mathbb{R}\]

 \2 Show that $\textrm{f}\textrm{f}\textrm{f}(x)=x$.\hfill[2]
 \begin{answer}
  \begin{align*}
   \textrm{f}\textrm{f}\textrm{f}(x) & =\textrm{f}\textrm{f}\left(\frac{1}{1-x}\right)    \\
                                     & = \textrm{f}\left(\frac{1}{1-\frac{1}{1-x}}\right) \\
                                     & = \textrm{f}\left(\frac{1-x}{1-x-1}\right)         \\
                                     & = \textrm{f}\left(-\frac{1}{x}+1\right)            \\
                                     & = -\frac{1}{(\frac{1}{1-x})}+1                     \\
                                     & = x\;\textrm{(shown)}
  \end{align*}
 \end{answer}
 \2 Show that $\textrm{f}^{-1}\textrm{f}^{-1}(x)=\textrm{f}(x)$.\hfill[2]
 \begin{answer}
  To find $\textrm{f}^{-1}(x)$, let us make $x$ the subject of $y=\frac{1}{1-x}$
  \begin{align*}
   y   & = \frac{1}{1-x}   \\
   1-x & = \frac{1}{y}     \\
   x   & = 1 - \frac{1}{y} \\
  \end{align*}
  Therefore, $\textrm{f}^{-1}(x) = 1 - \frac{1}{x}$.
 \end{answer}
 \begin{answer}
  We now simplify the composite function $\textrm{f}^{-1}\textrm{f}^{-1}(x).$
  \begin{align*}
   \textrm{f}^{-1}\textrm{f}^{-1}(x) & = \textrm{f}^{-1}(1 - \frac{1}{x}) \\
                                     & = 1 - \frac{1}{1 - \frac{1}{x}}    \\
                                     & = 1 - \frac{x}{x-1}                \\
                                     & = \frac{x-1-x}{x-1}                \\
                                     & = \frac{1}{1-x}                    \\
                                     & = \textrm{f}(x)\;\textrm{(shown)}
  \end{align*}
 \end{answer}
 \2 Hence, find the exact value(s) of a such that $\textrm{f}(\textrm{f}\textrm{f}\textrm{f}(a)+\textrm{f}\textrm{f}(a)+\textrm{f}(a))=\textrm{f}^{-1}(a)$. \hfill[4]
 \begin{answer}
  We will apply $\textrm{f}^{-1}$ on both sides so the RHS of the equation becomes $\textrm{f}^{-1}\textrm{f}^{-1}(a)$ to utilise our result in \textbf{(b)}. We then simplify the composite functions $\textrm{f}\textrm{f}\textrm{f}$ and $\textrm{f}\textrm{f}$ using our working in \textbf{(a)}.
  \begin{align*}
   \textrm{f}(\textrm{f}\textrm{f}\textrm{f}(a)+\textrm{f}\textrm{f}(a)+\textrm{f}(a)) & =\textrm{f}^{-1}(a)                \\
   \textrm{f}\textrm{f}\textrm{f}(a)+\textrm{f}\textrm{f}(a)+\textrm{f}(a)             & =\textrm{f}^{-1}\textrm{f}^{-1}(a) \\
   a-\frac{1}{a}+\textrm{f}(a)                                                         & =\textrm{f}(a)                     \\
   a^2-1                                                                               & =0                                 \\
   \therefore a                                                                        & =\pm 1
  \end{align*}
 \end{answer}
 %Question 4

 \1 The function f is defined by
 \[\textrm{f}(x)=
  \begin{cases}
   -\dfrac{1}{1+x} & \textrm{for }-1<x<1, \\
   \frac{1}{2}x-1  & \textrm{for }x\geq1.
  \end{cases}
 \] %Question 5

 \2 Sketch the graph of f. Hence, show that f is one-one. \hfill[3]
 \begin{answer}
  \vspace{3mm}
  \color{black}
  \begin{tikzpicture}
   \begin{axis}[
     axis lines = center,
     xlabel = $x$,
     ylabel = $y$,
     xmin = -2, xmax=5,
     ymin=-5, ymax = 2,
     legend pos=outer north east
    ]
    \addplot [
     domain=-1:1,
     samples=100,
     color=blue,
    ]
    {-1/(1+x)};
    \addplot [
     domain=1:5,
     samples=100,
     color=blue,
    ]
    {0.5*x-1};
    \addlegendentry{$\textrm{f}(x)$};
    \node[label={270:{$(1,-\frac{1}{2})$}},circle,fill,inner sep=2pt] at (axis cs:1,-0.5) {};
    \draw[dashed] ({axis cs:-1,0}|-{rel axis cs:0,0}) -- ({axis cs:-1,0}|-{rel axis cs:0,1});
   \end{axis}
  \end{tikzpicture}\\
  \color{blue}
  The graph of $y=a$ cuts $\textrm{f}(x)$ at one and only one point $\forall a\in\mathbb{R}\implies$ the function is one-one.
 \end{answer}

 \2 Find $\textrm{f}^{-1}$. \hfill[2]
 \begin{answer}
  Making $x$ the subject for both $y=-\dfrac{1}{1+x}$ and $y=\dfrac{1}{2}x-1$:
  \begin{align*}
   \implies x=-\frac{1}{y}-1 \quad\textrm{and}\quad x=2(y+1) \\
   \therefore\textrm{f}^{-1}(x)=
   \begin{cases}
    -\frac{1}{x}-1 & \textrm{for }x<-\frac{1}{2},    \\
    2(x+1)         & \textrm{for }x\geq-\frac{1}{2}.
   \end{cases}
  \end{align*}
 \end{answer}

 \1 The function f is defined by \[\textrm{f}:x \mapsto \frac{x^2+1}{x-1},\;x>a.\] %Question 6

 \2 Given that f is one-one, state the exact value of $a$. \hfill[2]
 \begin{answer}
  By plotting the result on a G.C., we can see that $a$ must be the $x$-coordinate of the minimum point of the curve. To obtain stationary points, we have to differentiate \textrm{f}$(x)$ with respect to $x$.
  \begin{align*}
   \textrm{f}(x)  & =x+1+\frac{2}{x-1}                                       \\
   \textrm{f}'(x) & =1-\frac{2}{(x-1)^2}=0 \textrm{   for stationary points} \\
                  & \implies (x-1)^2=2                                       \\
                  & x=1\pm \sqrt{2}
  \end{align*}
  From the G.C., we can observe that the $x$-coordinate of the minimum point is greater than that of the maximum point. Therefore, $a=1+\sqrt{2}$.
 \end{answer}
 \2 Find $\textrm{f}^{-1}(x)$ and state the domain and range of $\textrm{f}^{-1}$.\hfill[5]
 \begin{answer}
  To find $\textrm{f}^{-1}(x)$, let us make $x$ the subject of $y=\frac{x^2+1}{x-1}$.
  \begin{align*}
   y          & =\frac{x^2+1}{x-1}               \\
   y(x-1)     & =x^2+1                           \\
   x^2-yx+y+1 & =0                               \\
   x          & = \frac{y\pm \sqrt{y^2-4y-4}}{2}
  \end{align*}
  To eliminate the $\pm$, we must find the range of values of $y$, or $\textrm{R}_{\textrm{f}}$. From the G.C. we know that the $y$-coordinate of the minimum point is the lower limit of $\textrm{R}_{\textrm{f}}$.
  \begin{align*}
   \textrm{Substituting }x=1+\sqrt{2},\; y & =\frac{(1+\sqrt{2})^2+1}{1+\sqrt{2}-1} \\
                                           & = \frac{(4+2\sqrt{2})}{\sqrt{2}}       \\
                                           & = 2\sqrt{2}+2
  \end{align*}
  Since $y>2\sqrt{2}+2$, $\sqrt{y^2-4y-4}>\sqrt{(2\sqrt{2}+2)^2-4(2\sqrt{2}+2)-4}=0$.\\
  It follows that $\frac{y-\sqrt{y^2-4y-4}}{2}<1+\sqrt{2}$.\\
  However, $x \in (1+\sqrt{2},\infty)$, so we can only take $\frac{y+\sqrt{y^2-4y-4}}{2}$.\\
  $\therefore$\; $\textrm{f}^{-1}(x)=\frac{x+\sqrt{x^2-4x-4}}{2}$.\\
  As found earlier, $\textrm{D}_{\textrm{f}^{-1}}=\textrm{R}_{\textrm{f}}=(2+2\sqrt{2},\infty)$ and $\textrm{R}_{\textrm{f}^{-1}}=\textrm{D}_{\textrm{f}}=(1+\sqrt{2},\infty)$
 \end{answer}
 \1 Functions f and g are defined as follows:
 \begin{align*}
   & \textrm{f}(x)=\frac{1}{2}\mathrm{e}^{1-x},\;x\geq0, \\
   & \textrm{g}(x)=1-\ln{(x)},\;0<x\leq{\mathrm{e}}.
 \end{align*} %Question 7

 \2 Show that both fg and gf exist.\hfill[3]
 \begin{answer}
  \vspace{3mm}
  \color{black}
  \begin{tikzpicture}
   \begin{axis}[
     axis lines = center,
     xlabel = $x$,
     ylabel = $y$,
     xmin = 0, xmax=4,
     ymin=0, ymax = 4,
     legend pos=outer north east
    ]
    \addplot [
     domain=0:4,
     samples=100,
     color=blue,
    ]
    {0.5*e^(1-x)};
    \addlegendentry{$\textrm{f}(x)$};
    \addplot [
     domain=0:e,
     samples=100,
     color=red,
    ]
    {1-ln(x)};
    \addlegendentry{$\textrm{g}(x)$};
   \end{axis}
  \end{tikzpicture}\\
  \color{blue}
  R$_\textrm{g} = \mathbb{R}^+_0 =$ D$_\textrm{f}\implies$ fg exists \\
  R$_\textrm{f} = (0,\frac{1}{2}\mathrm{e}] \subset (0,\mathrm{e}]$ = D$_\textrm{g}\implies$ gf exists
 \end{answer}

 \2 By finding expressions for fg and gf, find the exact solution of $\textrm{fg}(2)=\textrm{gf}(\ln{(x)})$.\hfill[3]\\
 \begin{answer}
  fg$(x)=\frac{1}{2}\mathrm{e}^{1-(1-\ln{x})} = \frac{1}{2}x $ \\
  gf$(x) = 1-\ln{(\frac{1}{2}\mathrm{e}^{1-x}}) = 1-\ln{(\frac{1}{2}\cdot \mathrm{e} \cdot \mathrm{e}^{-x})} = 1-(-\ln2+1-x) = x+\ln2$
  \begin{align*}
   \textrm{gf}(\ln x) & = \ln2x              \\
                      & = \textrm{fg}(2) = 1 \\
   \implies x=\frac{\mathrm{e}}{2}
  \end{align*}
 \end{answer}

 \1 The functions f and g are defined by: \[\textrm{f}(x)=
  \begin{cases}
   2\sqrt{x-2}           & \textrm{for }2\leq x<6, \\
   6-\sqrt{\frac{2x}{3}} & \textrm{for }x\geq6.
  \end{cases}
  \;\;\;\;\textrm{and}\;\;\;\; \textrm{g}(x)=x^2,\;x\in \mathbb{R}.\]
 %Question 8

 \2 Show that gf exists. \hfill[2]
 \begin{answer}
  Notice that the maximum value of $\textrm{(x)}$ is equal to 4 when $x=6$.\\
  R$_\textrm{f} = (-\infty,4] \subset \mathbb{R} = \textrm{D}_{\textrm{g}}\implies$ gf exists.
 \end{answer}
 \2 Find the exact value of $\textrm{gf}(4)$.\hfill[2]
 \begin{answer}
  Since $4\in [2,6)$ for which $\textrm{f}(x)=2\sqrt{x-2}$,
  \begin{align*}
   \textrm{gf}(x) & = \textrm{g}(2\sqrt{x-2}) \\
                  & = (2\sqrt{x-2})^2         \\
                  & = 4(x-2)
  \end{align*}
  $\therefore \textrm{gf}(4)=4(2)=8$
 \end{answer}
 \2 Find the exact value of $x$ such that $\textrm{gf}(x)=5$.\hfill[2]
 \begin{answer}
  We have already found $\textrm{gf}(x)$ in terms of $x$ in the previous answer. All we need to do is to equate this expression to 5.
  \begin{align*}
   \textrm{gf}(x) = 4(x-2) & = 5           \\
   x                       & =\frac{13}{4}
  \end{align*}
 \end{answer}
 \1 The function h is defined by: \[\textrm{h}(n)=
  \begin{cases}
   n(\textrm{h}(n-1)) & \textrm{for }n>1, \\
   1                  & \textrm{for }n=1.
  \end{cases}\]
 where $n \in \mathbb{Z}^+$.

 \2 State the values of $\textrm{h}(2)$, $\textrm{h}(3)$ and $\textrm{h}(4)$.\hfill[1]
 \begin{answer}
  $h(2)=2(h(1))=2, h(3)=3(h(2))=6, h(4)=4(h(3))=24$
 \end{answer}

 \2 Deduce the use of function h.\hfill[1]
 \begin{answer}
  $h(n)=n!$
 \end{answer}

 \2 Evaluate $\dfrac{\textrm{h}(40)}{\textrm{h}(10)\textrm{h}(30)}$. \hfill[2]
 \begin{answer}
  Notice that this is just $\left(\begin{smallmatrix}40\\30\end{smallmatrix}\right)$.
  \begin{align*}
   \frac{\textrm{h}(40)}{\textrm{h}(10)\textrm{h}(30)} = \frac{40!}{10!30!} = 847660528
  \end{align*}
 \end{answer}

 \1 The function f has an inverse and is such that \[\textrm{f}:x^2+3 \mapsto x,\;x>0.\]

 \2 Find $\textrm{f}(x)$, and write down its domain and range.\hfill[3]
 \begin{answer}
  Making $x$ the subject of $y=x^2+3$, we get $x=\pm\sqrt{y-3}$. Since D$_{\textrm{f}^{-1}} = \mathbb{R}^+$, $f(x) = \sqrt{x-3}$ and R$_\textrm{f}=\mathbb{R}^+$ and D$_\textrm{f}=(3,\infty)$
 \end{answer}

 \2 The function g is such that $\textrm{g}(3x+2)=\textrm{f}(x)$.\\ Find $\textrm{g}(x)$. State its domain and range.\hfill[4]
 \begin{answer}
  Given that $\textrm{g}(3x+2)$ has exactly the same graph as $\textrm{f}(x)$, we can manipulate the terms within the brackets.\\
  Subtracting 2 units on  both sides, $\textrm{g}(3x)=\textrm{f}(x-2)$.\\
  Dividing by 3 on both sides,
  \begin{align*}
   \textrm{g}(x) & =\textrm{f}\left(\frac{x-2}{3}\right) \\
                 & = \sqrt{\frac{x-2}{3}-3}              \\
                 & = \sqrt{\frac{x-11}{3}}
  \end{align*}
  To find the domain of g, we perform the same transformations.\\
  Since $x>3$, $3x+2>11$. D$_{\textrm{g}}=(11,\infty)$\\
  The range of g is clearly R$_\textrm{g}=\mathbb{R}^+$.\\\\
  Alternatively, one can deduce this using the fact that $\textrm{g}(x)$ is a square root function, which means it cannot take negative values.
 \end{answer}

\end{outline}

\end{document}
